\documentclass[12pt,a4paper]{article}

\usepackage{fancyvrb}
\usepackage{hyperref}


\author{Cindy Lamm}
\title{Solution Abstract Problem 1: SmartFly}

\begin{document}
\VerbatimFootnotes
\maketitle

Explain your methodology including approach, assumptions, software and algorithms used, testing and validation techniques applied, model selection criteria, and total time spent.

\subsection{Approach}

\subsubsection{Get to know the data}

As first step I do a basic structure and content analysis of the given data files on the command line in order to prepare loading the data into any tool/program\footnote{see \verb+problem1_smartfly/src/00_file_structure_analysis/file_structure_analysis.sh+ }. The main goal is to get a first impression of data quality: Do all lines of the csv file have the same number of fields? Are there any quotes within the fields that might need special treatment?

As second step I do an exploratory data analysis on the historic as well as the prediction data set. I analyse both sets to avoid potential issues arising when the prediction data is slightly (or strongly) different in structure or content from the historic data that will be used for training a model. 

I use the open source statistics software R\footnote{\url{http://www.r-project.org/}} for the exploratory data analysis since it has all necessary analysis functions and plotting tools already implemented and the knitr package\footnote{\url{http://yihui.name/knitr/}} allows for reproducible analysis.


\end{document}