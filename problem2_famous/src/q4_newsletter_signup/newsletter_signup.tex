\documentclass{article}\usepackage[]{graphicx}\usepackage[]{color}
%% maxwidth is the original width if it is less than linewidth
%% otherwise use linewidth (to make sure the graphics do not exceed the margin)
\makeatletter
\def\maxwidth{ %
  \ifdim\Gin@nat@width>\linewidth
    \linewidth
  \else
    \Gin@nat@width
  \fi
}
\makeatother

\definecolor{fgcolor}{rgb}{0.345, 0.345, 0.345}
\newcommand{\hlnum}[1]{\textcolor[rgb]{0.686,0.059,0.569}{#1}}%
\newcommand{\hlstr}[1]{\textcolor[rgb]{0.192,0.494,0.8}{#1}}%
\newcommand{\hlcom}[1]{\textcolor[rgb]{0.678,0.584,0.686}{\textit{#1}}}%
\newcommand{\hlopt}[1]{\textcolor[rgb]{0,0,0}{#1}}%
\newcommand{\hlstd}[1]{\textcolor[rgb]{0.345,0.345,0.345}{#1}}%
\newcommand{\hlkwa}[1]{\textcolor[rgb]{0.161,0.373,0.58}{\textbf{#1}}}%
\newcommand{\hlkwb}[1]{\textcolor[rgb]{0.69,0.353,0.396}{#1}}%
\newcommand{\hlkwc}[1]{\textcolor[rgb]{0.333,0.667,0.333}{#1}}%
\newcommand{\hlkwd}[1]{\textcolor[rgb]{0.737,0.353,0.396}{\textbf{#1}}}%

\usepackage{framed}
\makeatletter
\newenvironment{kframe}{%
 \def\at@end@of@kframe{}%
 \ifinner\ifhmode%
  \def\at@end@of@kframe{\end{minipage}}%
  \begin{minipage}{\columnwidth}%
 \fi\fi%
 \def\FrameCommand##1{\hskip\@totalleftmargin \hskip-\fboxsep
 \colorbox{shadecolor}{##1}\hskip-\fboxsep
     % There is no \\@totalrightmargin, so:
     \hskip-\linewidth \hskip-\@totalleftmargin \hskip\columnwidth}%
 \MakeFramed {\advance\hsize-\width
   \@totalleftmargin\z@ \linewidth\hsize
   \@setminipage}}%
 {\par\unskip\endMakeFramed%
 \at@end@of@kframe}
\makeatother

\definecolor{shadecolor}{rgb}{.97, .97, .97}
\definecolor{messagecolor}{rgb}{0, 0, 0}
\definecolor{warningcolor}{rgb}{1, 0, 1}
\definecolor{errorcolor}{rgb}{1, 0, 0}
\newenvironment{knitrout}{}{} % an empty environment to be redefined in TeX

\usepackage{alltt}
\usepackage[sc]{mathpazo}
\usepackage[T1]{fontenc}
\usepackage{geometry}
\geometry{verbose,tmargin=2.5cm,bmargin=2.5cm,lmargin=2.5cm,rmargin=2.5cm}
\usepackage{fullpage}
\usepackage{pdflscape}
\setcounter{secnumdepth}{2}
\setcounter{tocdepth}{2}
\usepackage{datetime}
\usepackage{url}
\usepackage{listings}
\usepackage[unicode=true,pdfusetitle,
 bookmarks=true,bookmarksnumbered=true,bookmarksopen=true,bookmarksopenlevel=2,
 breaklinks=false,pdfborder={0 0 1},backref=false,colorlinks=false]
 {hyperref}
\hypersetup{
 pdfstartview={XYZ null null 1}}
\IfFileExists{upquote.sty}{\usepackage{upquote}}{}
\begin{document}
\flushbottom



\date{\currenttime, \today}
\title{Almost Famous: Analyse Newsletter Signup Rate Per Experiment}
\author{Cindy Lamm}

\maketitle

% ------------------------
% ------------------------ define functions
% ------------------------





% ------------------------
% ------------------------ web analysis starts here
% ------------------------

Load variable names and types:
\begin{knitrout}
\definecolor{shadecolor}{rgb}{0.969, 0.969, 0.969}\color{fgcolor}\begin{kframe}
\begin{alltt}
\hlstd{nameTypeDataFile}  \hlkwb{<-} \hlstr{"../../data/raw_variables.csv"}
\hlstd{variableNames} \hlkwb{<-} \hlkwd{read.csv}\hlstd{(nameTypeDataFile,} \hlkwc{header}\hlstd{=}\hlnum{TRUE}\hlstd{,} \hlkwc{stringsAsFactors}\hlstd{=}\hlnum{FALSE}\hlstd{)}
\hlstd{variableNames}
\end{alltt}
\begin{verbatim}
##          name      type
## 1    visit_id    factor
## 2         uid    factor
## 3    campaign    factor
## 4      tstamp character
## 5 experiments    factor
## 6      action    factor
## 7       query    factor
\end{verbatim}
\begin{alltt}
\hlstd{factorIdx} \hlkwb{<-} \hlkwd{which}\hlstd{(variableNames}\hlopt{$}\hlstd{type}\hlopt{==}\hlstr{"factor"}\hlstd{)}
\hlstd{factorNames} \hlkwb{<-} \hlstd{variableNames}\hlopt{$}\hlstd{name[factorIdx]}
\end{alltt}
\end{kframe}
\end{knitrout}

Read the per visit aggregated web log data:
\begin{knitrout}
\definecolor{shadecolor}{rgb}{0.969, 0.969, 0.969}\color{fgcolor}\begin{kframe}
\begin{alltt}
\hlstd{visitFile} \hlkwb{<-} \hlstr{"../../data/web_visits.csv"}
\hlcom{# visitFile <- "../../data/head2000.csv"}
\hlstd{visitData} \hlkwb{<-} \hlkwd{read.csv}\hlstd{(visitFile,} \hlkwc{stringsAsFactors}\hlstd{=}\hlnum{FALSE}\hlstd{,} \hlkwc{col.names}\hlstd{=variableNames}\hlopt{$}\hlstd{name,}
                      \hlkwc{colClasses}\hlstd{=variableNames}\hlopt{$}\hlstd{type,} \hlkwc{na.strings}\hlstd{=}\hlkwd{c}\hlstd{(}\hlstr{"NA"}\hlstd{,}\hlstr{""}\hlstd{))}
\hlstd{visitData}\hlopt{$}\hlstd{tstamp} \hlkwb{<-} \hlkwd{as.POSIXct}\hlstd{(visitData}\hlopt{$}\hlstd{tstamp)}
\hlkwd{str}\hlstd{(visitData)}
\end{alltt}
\begin{verbatim}
## 'data.frame':	1482602 obs. of  7 variables:
##  $ visit_id   : Factor w/ 1482602 levels "10000024498",..: 1252062 128641 583195 349394 830165 690964 251984 106944 1297398 803992 ...
##  $ uid        : Factor w/ 1064214 levels "100000493","100000682",..: 858988 92339 95584 929716 656934 710857 223903 725904 857529 562160 ...
##  $ campaign   : Factor w/ 10 levels "103","127","14",..: 7 1 7 1 4 8 1 1 1 2 ...
##  $ tstamp     : POSIXct, format: "2014-09-18 05:43:18" "2014-09-16 21:24:08" ...
##  $ experiments: Factor w/ 4 levels "[1 3]","[1 4]",..: 2 1 4 1 3 2 1 3 2 1 ...
##  $ action     : Factor w/ 8 levels "[landed adclick adclick adclick]",..: 8 8 8 8 8 8 8 8 8 8 ...
##  $ query      : Factor w/ 5 levels "advanced analytics",..: 1 5 1 5 3 5 5 5 5 4 ...
\end{verbatim}
\end{kframe}
\end{knitrout}

\pagebreak 

\begin{knitrout}
\definecolor{shadecolor}{rgb}{0.969, 0.969, 0.969}\color{fgcolor}\begin{kframe}
\begin{alltt}
\hlkwd{summary}\hlstd{(visitData)}
\end{alltt}
\begin{verbatim}
##         visit_id              uid             campaign          tstamp                   
##  10000024498:      1   102486699:      7   558    :324872   Min.   :2014-09-15 00:00:01  
##  10000032484:      1   123618732:      7   103    :324027   1st Qu.:2014-09-18 16:32:04  
##  10000079220:      1   143588980:      7   59     :232002   Median :2014-09-22 16:55:36  
##  10000092303:      1   159226004:      7   31     :231685   Mean   :2014-09-22 20:33:11  
##  10000132469:      1   168873739:      7   127    : 92681   3rd Qu.:2014-09-26 19:41:15  
##  10000206890:      1   171898393:      7   94     : 92436   Max.   :2014-09-30 23:53:20  
##  (Other)    :1482596   (Other)  :1482560   (Other):184899                                
##  experiments                                 action       
##  [1 3]:370018   landed                          :1291256  
##  [1 4]:371852   [landed signup]                 :  84889  
##  [2 3]:370082   [landed order]                  :  43930  
##  [2 4]:370650   [landed adclick]                :  28233  
##                 [landed adclick adclick adclick]:  14956  
##                 [landed adclick adclick]        :  14875  
##                 (Other)                         :   4463  
##                         query       
##  advanced analytics        :463687  
##  building predictive models: 92454  
##  data science              : 92445  
##  data science training     :185117  
##  predictive modeling       :648899  
##                                     
## 
\end{verbatim}
\end{kframe}
\end{knitrout}

\section{Newsletter Signup Rate Per Experiment}

\begin{quote}
Compute the overall newsletter signup rate (defined as the number of users who signed up to the newsletter divided by the total number of users) for each of the experiments.
\end{quote}


What are the actions per visit?
\begin{knitrout}
\definecolor{shadecolor}{rgb}{0.969, 0.969, 0.969}\color{fgcolor}\begin{kframe}
\begin{alltt}
\hlkwd{table}\hlstd{(visitData}\hlopt{$}\hlstd{action)}
\end{alltt}
\begin{verbatim}
## 
## [landed adclick adclick adclick]         [landed adclick adclick] 
##                            14956                            14875 
##                 [landed adclick]                   [landed order] 
##                            28233                            43930 
##          [landed signup adclick]            [landed signup order] 
##                             1045                             3418 
##                  [landed signup]                           landed 
##                            84889                          1291256
\end{verbatim}
\end{kframe}
\end{knitrout}

Look at visits with signups:
\begin{knitrout}
\definecolor{shadecolor}{rgb}{0.969, 0.969, 0.969}\color{fgcolor}\begin{kframe}
\begin{alltt}
\hlstd{signupIdx} \hlkwb{<-} \hlkwd{getPatternIndex}\hlstd{(visitData}\hlopt{$}\hlstd{action,} \hlstr{"signup"}\hlstd{)}
\end{alltt}
\begin{verbatim}
## Concerned pattern levels are [landed signup adclick], [landed signup order], [landed signup]
\end{verbatim}
\begin{alltt}
\hlstd{totalSignups} \hlkwb{<-} \hlkwd{length}\hlstd{(signupIdx)}
\end{alltt}
\end{kframe}
\end{knitrout}

I conclude from the factor levels for \verb+action+ that there is at most 1 signup per visit and overall 89352 signups. I cross check with a simple grep on the command line on the unaggregated web data which gives us the same result:

\begin{lstlisting}
$ grep -o signup web.log | wc -l
$ 89352
\end{lstlisting}

Add the number of signups per visit as variable to the data frame:
\begin{knitrout}
\definecolor{shadecolor}{rgb}{0.969, 0.969, 0.969}\color{fgcolor}\begin{kframe}
\begin{alltt}
\hlstd{nbSignup} \hlkwb{<-} \hlkwd{rep}\hlstd{(}\hlnum{0}\hlstd{,} \hlkwd{nrow}\hlstd{(visitData))}
\hlstd{nbSignup[signupIdx]} \hlkwb{<-} \hlnum{1}
\hlstd{visitData}\hlopt{$}\hlstd{nb_signups} \hlkwb{<-} \hlstd{nbSignup}
\end{alltt}
\end{kframe}
\end{knitrout}



There are $93.97\%$ of visits that don't have a signup and only $6.03\%$ that do.

Checkout experiment information:
\begin{knitrout}
\definecolor{shadecolor}{rgb}{0.969, 0.969, 0.969}\color{fgcolor}\begin{kframe}
\begin{alltt}
\hlkwd{prop.table}\hlstd{(}\hlkwd{table}\hlstd{(visitData}\hlopt{$}\hlstd{experiments))}
\end{alltt}
\begin{verbatim}
## 
##     [1 3]     [1 4]     [2 3]     [2 4] 
## 0.2495734 0.2508104 0.2496166 0.2499997
\end{verbatim}
\end{kframe}
\end{knitrout}

Split up the experiment information into separate variables
\begin{knitrout}
\definecolor{shadecolor}{rgb}{0.969, 0.969, 0.969}\color{fgcolor}\begin{kframe}
\begin{alltt}
\hlstd{expIdx1} \hlkwb{<-} \hlkwd{getPatternIndex}\hlstd{(visitData}\hlopt{$}\hlstd{experiments,} \hlnum{1}\hlstd{)}
\end{alltt}
\begin{verbatim}
## Concerned pattern levels are [1 3], [1 4]
\end{verbatim}
\begin{alltt}
\hlstd{totalExp1} \hlkwb{<-} \hlkwd{length}\hlstd{(expIdx1)}
\hlstd{expIdx2} \hlkwb{<-} \hlkwd{getPatternIndex}\hlstd{(visitData}\hlopt{$}\hlstd{experiments,} \hlnum{2}\hlstd{)}
\end{alltt}
\begin{verbatim}
## Concerned pattern levels are [2 3], [2 4]
\end{verbatim}
\begin{alltt}
\hlstd{totalExp2} \hlkwb{<-} \hlkwd{length}\hlstd{(expIdx2)}
\hlstd{expIdx3} \hlkwb{<-} \hlkwd{getPatternIndex}\hlstd{(visitData}\hlopt{$}\hlstd{experiments,} \hlnum{3}\hlstd{)}
\end{alltt}
\begin{verbatim}
## Concerned pattern levels are [1 3], [2 3]
\end{verbatim}
\begin{alltt}
\hlstd{totalExp3} \hlkwb{<-} \hlkwd{length}\hlstd{(expIdx3)}
\hlstd{expIdx4} \hlkwb{<-} \hlkwd{getPatternIndex}\hlstd{(visitData}\hlopt{$}\hlstd{experiments,} \hlnum{4}\hlstd{)}
\end{alltt}
\begin{verbatim}
## Concerned pattern levels are [1 4], [2 4]
\end{verbatim}
\begin{alltt}
\hlstd{totalExp4} \hlkwb{<-} \hlkwd{length}\hlstd{(expIdx4)}
\end{alltt}
\end{kframe}
\end{knitrout}

and add them pairwise to the data frame:
\begin{knitrout}
\definecolor{shadecolor}{rgb}{0.969, 0.969, 0.969}\color{fgcolor}\begin{kframe}
\begin{alltt}
\hlkwd{stopifnot}\hlstd{(}\hlopt{!}\hlkwd{any}\hlstd{(}\hlkwd{intersect}\hlstd{(expIdx1, expIdx2)),}
          \hlstd{totalExp1} \hlopt{+} \hlstd{totalExp2} \hlopt{==} \hlkwd{nrow}\hlstd{(visitData),}
          \hlopt{!}\hlkwd{any}\hlstd{(}\hlkwd{intersect}\hlstd{(expIdx3, expIdx4)),}
          \hlstd{totalExp3} \hlopt{+} \hlstd{totalExp4} \hlopt{==} \hlkwd{nrow}\hlstd{(visitData))}

\hlstd{experiment12} \hlkwb{<-} \hlkwd{rep}\hlstd{(}\hlnum{1}\hlstd{,} \hlkwd{nrow}\hlstd{(visitData))}
\hlstd{experiment12[expIdx2]} \hlkwb{<-} \hlnum{2}
\hlstd{visitData}\hlopt{$}\hlstd{experiment_12} \hlkwb{<-} \hlkwd{factor}\hlstd{(experiment12,} \hlkwc{levels}\hlstd{=}\hlnum{1}\hlopt{:}\hlnum{2}\hlstd{)}

\hlstd{experiment34} \hlkwb{<-} \hlkwd{rep}\hlstd{(}\hlnum{3}\hlstd{,} \hlkwd{nrow}\hlstd{(visitData))}
\hlstd{experiment34[expIdx4]} \hlkwb{<-} \hlnum{4}
\hlstd{visitData}\hlopt{$}\hlstd{experiment_34} \hlkwb{<-} \hlkwd{factor}\hlstd{(experiment34,} \hlkwc{levels}\hlstd{=}\hlnum{3}\hlopt{:}\hlnum{4}\hlstd{)}
\end{alltt}
\end{kframe}
\end{knitrout}

Checkout experiment distribution:
\begin{knitrout}
\definecolor{shadecolor}{rgb}{0.969, 0.969, 0.969}\color{fgcolor}\begin{kframe}
\begin{alltt}
\hlkwd{prop.table}\hlstd{(}\hlkwd{table}\hlstd{(visitData}\hlopt{$}\hlstd{experiment_12))}
\end{alltt}
\begin{verbatim}
## 
##         1         2 
## 0.5003838 0.4996162
\end{verbatim}
\begin{alltt}
\hlkwd{prop.table}\hlstd{(}\hlkwd{table}\hlstd{(visitData}\hlopt{$}\hlstd{experiment_34))}
\end{alltt}
\begin{verbatim}
## 
##         3         4 
## 0.4991899 0.5008101
\end{verbatim}
\end{kframe}
\end{knitrout}


How many signups are there per experiment?
\begin{knitrout}
\definecolor{shadecolor}{rgb}{0.969, 0.969, 0.969}\color{fgcolor}\begin{kframe}
\begin{alltt}
\hlstd{visitAggExp12} \hlkwb{<-} \hlkwd{aggregatePerExperiment12}\hlstd{(visitData)}
\hlstd{visitAggExp12}
\end{alltt}
\begin{verbatim}
##   experiment_12 nb_visits nb_uids total_signups total_non_signups signup_rate
## 1             1    741870  532225         45145            487080  0.08482315
## 2             2    740732  531989         44207            487782  0.08309758
\end{verbatim}
\begin{alltt}
\hlstd{visitAggExp34} \hlkwb{<-} \hlkwd{aggregatePerExperiment34}\hlstd{(visitData)}
\hlstd{visitAggExp34}
\end{alltt}
\begin{verbatim}
##   experiment_34 nb_visits nb_uids total_signups total_non_signups signup_rate
## 1             3    740100  531345         46819            484526  0.08811413
## 2             4    742502  532869         42533            490336  0.07981887
\end{verbatim}
\end{kframe}
\end{knitrout}

Write the result into file \verb+overall_signup_rate.json+:
\begin{knitrout}
\definecolor{shadecolor}{rgb}{0.969, 0.969, 0.969}\color{fgcolor}\begin{kframe}
\begin{alltt}
\hlkwd{library}\hlstd{(jsonlite)}
\hlstd{overallSignupRates} \hlkwb{<-} \hlkwd{c}\hlstd{(visitAggExp12}\hlopt{$}\hlstd{signup_rate, visitAggExp34}\hlopt{$}\hlstd{signup_rate)}
\hlkwd{names}\hlstd{(overallSignupRates)} \hlkwb{<-} \hlkwd{paste}\hlstd{(}\hlstr{"experiment"}\hlstd{,} \hlnum{1}\hlopt{:}\hlnum{4}\hlstd{,} \hlkwc{sep}\hlstd{=}\hlstr{""}\hlstd{)}
\hlstd{jsonString} \hlkwb{<-} \hlkwd{toJSON}\hlstd{(}\hlkwd{as.data.frame}\hlstd{(}\hlkwd{t}\hlstd{(overallSignupRates)),} \hlkwc{dataframe}\hlstd{=}\hlstr{"rows"}\hlstd{,} \hlkwc{pretty}\hlstd{=}\hlnum{TRUE}\hlstd{)}
\hlstd{jsonString}
\end{alltt}
\begin{verbatim}
## [
##     {
##         "experiment1": 0.0848,
##         "experiment2": 0.0831,
##         "experiment3": 0.0881,
##         "experiment4": 0.0798
##     }
## ]
## 
\end{verbatim}
\begin{alltt}
\hlkwd{write}\hlstd{(jsonString,} \hlkwc{file}\hlstd{=}\hlstr{"../q4_newsletter_signup/out/overall_signup_rate.json"}\hlstd{)}
\end{alltt}
\end{kframe}
\end{knitrout}

\section{Performance Of Experiments}

\begin{quote}
Use a G-test to compare the performance of experiment one vs. experiment two and experiment three vs. experiment four overall.
\end{quote}

I assume that the performance of an experiment is measured by the number of signups for the newsletter. 
\begin{knitrout}
\definecolor{shadecolor}{rgb}{0.969, 0.969, 0.969}\color{fgcolor}\begin{kframe}
\begin{alltt}
\hlstd{gTestTable12} \hlkwb{<-} \hlkwd{cbind}\hlstd{(visitAggExp12}\hlopt{$}\hlstd{total_signups, visitAggExp12}\hlopt{$}\hlstd{total_non_signups)}
\hlkwd{rownames}\hlstd{(gTestTable12)} \hlkwb{<-} \hlkwd{paste}\hlstd{(}\hlstr{"Experiment"}\hlstd{,}\hlnum{1}\hlopt{:}\hlnum{2}\hlstd{,}\hlkwc{sep}\hlstd{=}\hlstr{""}\hlstd{)}
\hlkwd{colnames}\hlstd{(gTestTable12)} \hlkwb{<-} \hlkwd{c}\hlstd{(}\hlstr{"signups"}\hlstd{,}\hlstr{"ignorations"}\hlstd{)}
\hlstd{gTestTable12}
\end{alltt}
\begin{verbatim}
##             signups ignorations
## Experiment1   45145      487080
## Experiment2   44207      487782
\end{verbatim}
\begin{alltt}
\hlstd{gTestTable34} \hlkwb{<-} \hlkwd{cbind}\hlstd{(visitAggExp34}\hlopt{$}\hlstd{total_signups, visitAggExp34}\hlopt{$}\hlstd{total_non_signups)}
\hlkwd{rownames}\hlstd{(gTestTable34)} \hlkwb{<-} \hlkwd{paste}\hlstd{(}\hlstr{"Experiment"}\hlstd{,}\hlnum{3}\hlopt{:}\hlnum{4}\hlstd{,}\hlkwc{sep}\hlstd{=}\hlstr{""}\hlstd{)}
\hlkwd{colnames}\hlstd{(gTestTable34)} \hlkwb{<-} \hlkwd{c}\hlstd{(}\hlstr{"signups"}\hlstd{,}\hlstr{"ignorations"}\hlstd{)}
\hlstd{gTestTable34}
\end{alltt}
\begin{verbatim}
##             signups ignorations
## Experiment3   46819      484526
## Experiment4   42533      490336
\end{verbatim}
\end{kframe}
\end{knitrout}






Run G-Test, which is a test of independence (just like the Chi-Square test). The null hypothesis is that the two variables are independent, which is rejected if the p-value is smaller than a chosen significance level $\alpha$. Usually one uses $\alpha=5\%$ but with such a large sample size I would opt for $\alpha=0.1\%$, which means that if we decide to reject the null hypothesis and assume the variables are dependent (i.e. in this case this means one experiment is better than the other), we might be wrong for $\alpha \times n=1064$ observations.

Run G-Test for experiments 1 and 2:
\begin{knitrout}
\definecolor{shadecolor}{rgb}{0.969, 0.969, 0.969}\color{fgcolor}\begin{kframe}
\begin{alltt}
\hlkwd{library}\hlstd{(Deducer)}
\hlstd{test12} \hlkwb{<-} \hlkwd{likelihood.test}\hlstd{(gTestTable12)}
\hlkwd{print}\hlstd{(test12)}
\end{alltt}
\begin{verbatim}
## 
## 	Log likelihood ratio (G-test) test of independence without correction
## 
## data:  gTestTable12
## Log likelihood ratio statistic (G) = 10.3003, X-squared df = 1, p-value =
## 0.00133
\end{verbatim}
\begin{alltt}
\hlcom{# sampleSize}
\hlkwd{sum}\hlstd{(visitAggExp12}\hlopt{$}\hlstd{nb_uids)}
\end{alltt}
\begin{verbatim}
## [1] 1064214
\end{verbatim}
\end{kframe}
\end{knitrout}

Run G-Test for experiments 3 and 4:
\begin{knitrout}
\definecolor{shadecolor}{rgb}{0.969, 0.969, 0.969}\color{fgcolor}\begin{kframe}
\begin{alltt}
\hlstd{test34} \hlkwb{<-} \hlkwd{likelihood.test}\hlstd{(gTestTable34)}
\hlkwd{print}\hlstd{(test34)}
\end{alltt}
\begin{verbatim}
## 
## 	Log likelihood ratio (G-test) test of independence without correction
## 
## data:  gTestTable34
## Log likelihood ratio statistic (G) = 238.1123, X-squared df = 1, p-value <
## 2.2e-16
\end{verbatim}
\begin{alltt}
\hlcom{#sample size}
\hlkwd{sum}\hlstd{(visitAggExp34}\hlopt{$}\hlstd{nb_uids)}
\end{alltt}
\begin{verbatim}
## [1] 1064214
\end{verbatim}
\end{kframe}
\end{knitrout}

I would interpret the results as follows: 
\begin{itemize}
  \item With a p-value of $p_{12}=0.0013301$ experiments 1 and 2 are indistinguishable - it doesn't matter for the newsletter signup whether the color scheme of the page is blue or green.
	\item With a p-value of $p_{34}=0$ there is a statistically significant difference between experiment 3 and 4 - based on the number of newsletter signups the promotional blurb by Josh Willis performs better.	 
\end{itemize}

Write the result into file \verb+performance_of_experiment_1_versus_2.json+:
\begin{knitrout}
\definecolor{shadecolor}{rgb}{0.969, 0.969, 0.969}\color{fgcolor}\begin{kframe}
\begin{alltt}
\hlkwd{write}\hlstd{(test12}\hlopt{$}\hlstd{p.value,} \hlkwc{file}\hlstd{=}\hlstr{"../q4_newsletter_signup/out/performance_of_experiment_1_versus_2.json"}\hlstd{)}
\hlkwd{write}\hlstd{(test34}\hlopt{$}\hlstd{p.value,} \hlkwc{file}\hlstd{=}\hlstr{"../q4_newsletter_signup/out/performance_of_experiment_3_versus_4.json"}\hlstd{)}
\end{alltt}
\end{kframe}
\end{knitrout}


\section{Required Days for $99\%$ Confidence in G-Test Result}

\begin{quote}
How many full days of data, starting from the first day, are required to determine that the newsletter signup rate for experiment one is better than experiment two at the 99\% confidence level?
\end{quote}

The number of days is "encoded" within the number of visits and users. Reformulated the question is "Given the current signup rate for each experiment and $\alpha=1\%$, how long does an experiment need to run in order to generate enough users/visits for the G-Test to be significant?"

I simulate the answer\footnote{just I like did for proportion test: \url{https://github.com/clamm/R_dev/blob/master/ab_testing.R}}.

Define a function to get the p-value from a G-Test:
\begin{knitrout}
\definecolor{shadecolor}{rgb}{0.969, 0.969, 0.969}\color{fgcolor}\begin{kframe}
\begin{alltt}
\hlkwd{library}\hlstd{(Deducer)}
\hlstd{getPValue} \hlkwb{<-} \hlkwa{function}\hlstd{(}\hlkwc{signupRates}\hlstd{,} \hlkwc{nbUsers}\hlstd{) \{}
  \hlcom{# parameters:}
  \hlcom{#   signupRates:  vector of signup rate for experiments}
  \hlcom{#   nbUsers:      vector of number of users how saw the experiment}
  \hlcom{# returns:}
  \hlcom{#   p-value}
  \hlcom{# description:}
  \hlcom{#   H0: experiments perform equally well based on signup rate}
  \hlcom{#   accept H1 if pval < alpha (i.e. the 2 experiments perform differently) }
  \hlstd{dataExperiments} \hlkwb{<-} \hlkwd{c}\hlstd{(signupRates} \hlopt{*} \hlstd{nbUsers, (}\hlnum{1}\hlopt{-}\hlstd{signupRates)} \hlopt{*} \hlstd{nbUsers)}
  \hlstd{tableExperiments} \hlkwb{<-} \hlkwd{matrix}\hlstd{(dataExperiments,} \hlkwc{nrow}\hlstd{=}\hlnum{2}\hlstd{,} \hlkwc{byrow}\hlstd{=}\hlnum{FALSE}\hlstd{)}
  \hlkwd{rownames}\hlstd{(tableExperiments)} \hlkwb{<-} \hlkwd{paste}\hlstd{(}\hlstr{"Experiment"}\hlstd{,LETTERS[}\hlnum{1}\hlopt{:}\hlnum{2}\hlstd{],}\hlkwc{sep}\hlstd{=}\hlstr{" "}\hlstd{)}
  \hlkwd{colnames}\hlstd{(tableExperiments)} \hlkwb{<-} \hlkwd{c}\hlstd{(}\hlstr{"signups"}\hlstd{,}\hlstr{"ignorations"}\hlstd{)}
  \hlkwd{print}\hlstd{(tableExperiments)}
  \hlstd{res} \hlkwb{<-} \hlkwd{likelihood.test}\hlstd{(tableExperiments)}
  \hlkwd{return}\hlstd{(res}\hlopt{$}\hlstd{p.value)}
\hlstd{\}}
\end{alltt}
\end{kframe}
\end{knitrout}

Double check that this function gives the same results as manual extraction of the p-value:
\begin{knitrout}
\definecolor{shadecolor}{rgb}{0.969, 0.969, 0.969}\color{fgcolor}\begin{kframe}
\begin{alltt}
\hlstd{newPvalue} \hlkwb{<-} \hlkwd{getPValue}\hlstd{(overallSignupRates[}\hlnum{1}\hlopt{:}\hlnum{2}\hlstd{], visitAggExp12}\hlopt{$}\hlstd{nb_uids)}
\end{alltt}
\begin{verbatim}
##              signups ignorations
## Experiment A   45145      487080
## Experiment B   44207      487782
\end{verbatim}
\begin{alltt}
\hlstd{newPvalue} \hlopt{==} \hlstd{test12}\hlopt{$}\hlstd{p.value}
\end{alltt}
\begin{verbatim}
## p.value 
##    TRUE
\end{verbatim}
\end{kframe}
\end{knitrout}



\end{document}
